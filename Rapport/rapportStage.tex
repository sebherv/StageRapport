\documentclass[12pt,a4paper]{report}
\setcounter{secnumdepth}{5} % setting level of numbering (default for "report" is 3).
\usepackage{natbib}
\usepackage[utf8]{inputenc}
\usepackage[french]{babel}
\usepackage[T1]{fontenc}
\usepackage{amsmath}
\usepackage{amsfonts}
\usepackage{amssymb}
\usepackage{graphicx}
\usepackage{appendix}
\usepackage{enumitem}
\usepackage{float} 
\usepackage[nottoc, notlof, notlot]{tocbibind}
\usepackage{pdfpages}
\usepackage{gensymb}
\usepackage{lipsum}
\usepackage[colorinlistoftodos,textwidth=3.7cm]{todonotes}
\usepackage[top=1in, bottom=1in, left=1in, right=1in]{geometry}
\usepackage{tocbibind}
\usepackage{hyperref}
\author{Sébastien Hervieu}

\setcounter{MaxMatrixCols}{20}
\newcounter{mypagecount}% create a new counter
\setcounter{mypagecount}{0}% set it to something just in case
\newenvironment{interlude}{% create a new environment for the unnumbered section(s)
	\clearpage
	\setcounter{mypagecount}{\value{page}}% use the new counter we created to hold the page count at the start of the unnumbered section
	\thispagestyle{empty}% we want this page to be empty (adjust to use a modified page style)
	\pagestyle{empty}% use the same style for subsequent pages in the unnumbered section
}{%
	\clearpage
	\setcounter{page}{\value{mypagecount}}% restore the incremented value to the official tally of pages so the page numbering continues correctly
}


\begin{document}
	
\newcommand{\para}{\paragraph*{}}

\newcommand{\bs}[1]{\boldsymbol{#1}}
	
\newcommand{\todoRediger}[1]{\todo[color=red,inline]{REDIGER #1}}
\newcommand{\todoARevoir}[1]{\todo[color=red,inline]{A REVOIR #1}}
\newcommand{\todoCompleter}[1]{\todo[inline]{COMPLETER #1}}
\newcommand{\todoAmeliorer}[1]{\todo{AMELIORER: #1}}
\newcommand{\todoMiseEnForme}[1]{\todo[color=green,inline]{MISE EN FORME: #1}}
\newcommand{\todoCheck}[1]{\todo[color=cyan,fancyline]{VERIFIER #1}}
\newcommand{\todoAjouterRef}[1]{\todo[color=pink]{AJOUTER REF #1}}
\newcommand{\todoTrouverRefExt}[1]{\todo[color=red,inline]{TROUVER REF EXTERNE: #1}}
\newcommand{\todoObjectif}[1]{\todo[color=yellow,inline,size=\Large]{OBJECTIF: #1}}
\newcommand{\todoPeutEtre}[1]{\todo[color=teal,inline]{PEUT ETRE #1 ?}}

\begin{titlepage}

\newcommand{\HRule}{\rule{\linewidth}{0.5mm}} % Defines a new command for the horizontal lines, change thickness here

\center % Center everything on the page
 
%----------------------------------------------------------------------------------------
%	HEADING SECTIONS
%----------------------------------------------------------------------------------------

\textsc{\LARGE Université de Rennes 1}\\[1cm] 
\textsc{\Large }\\[0.5cm] % Major heading such as course name
\textsc{\large Master 2 Calcul scientifique et modélisation}\\
\textsc{Rapport de Stage}\\
%----------------------------------------------------------------------------------------
%	TITLE SECTION
%----------------------------------------------------------------------------------------

\HRule \\[0.4cm]
{ \huge \bfseries Etude et développement d’outils mathématiques pour estimer, en temps réel, le tassage et le volume d’un silo de maïs à partir de capteurs embarqués}\\[0.4cm] 
\HRule \\[1.5cm]
 
%----------------------------------------------------------------------------------------
%	AUTHOR SECTION
%----------------------------------------------------------------------------------------

\begin{minipage}{0.4\textwidth}
\begin{flushleft} \large
\emph{Auteur:}\\
Sébastien \textsc{Hervieu}
\end{flushleft}
\end{minipage}
~
\begin{minipage}{0.4\textwidth}

\begin{flushright}
	\emph{Tuteur de Stage:} \\
	Geoffroy \textsc{Etaix}
\end{flushright}

\begin{flushright} \large
	\emph{Tuteur Universitaire:} \\
	Fabrice \textsc{Mahé} 
\end{flushright}

\end{minipage}\\[1cm]

% If you don't want a supervisor, uncomment the two lines below and remove the section above
%\Large \emph{Author:}\\
%John \textsc{Smith}\\[3cm] % Your name

%----------------------------------------------------------------------------------------
%	DATE SECTION
%----------------------------------------------------------------------------------------

{\large \today}\\[1cm] % Date, change the \today to a set date if you want to be precise

%----------------------------------------------------------------------------------------
%	LOGO SECTION
%----------------------------------------------------------------------------------------
\includegraphics[height=3cm]{img/logo-tellusenv.png} \\
\includegraphics[height=3cm]{img/univ.jpeg}\\[1cm] % Include a department/university logo - this will require the graphicx package
 
%----------------------------------------------------------------------------------------

\vfill % Fill the rest of the page with whitespace

\end{titlepage}


\begin{interlude}
\listoftodos
\end{interlude}


\cleardoublepage% especially in a document where chapters start at right-hand pages
%\phantomsection% for an anchor if you use hyperref

\chapter*{Remerciements}% for the actuall unnumbered heading
\addcontentsline{toc}{chapter}{\numberline{}Remerciements}% if you wish to have a TOC entry
\thispagestyle{empty}% or plain etc.
\markboth{Remerciements}{Remerciements}% relevant depending on page style
% or if it's more than one page


Je tiens à remercier les personnes qui m'ont permis de prés ou de loin à accomplir ce stage et qui m'ont aidé lors de la rédaction de ce rapport.

Tout d'abord, j'adresse mes remerciements à Messieurs \textbf{Fabrice Mahé et Eric Darrigrand} profeseurs de l'Université de Rennes 1, qui m'ont permis de suivre cette formation et qui m'ont accompagné lors de la recherche de stage.

Je tiens à remercier Monsieur \textbf{Geoffroy ETAIX}, qui m'a accordé sa confiance pour le stage et pour etc 
\todoCompleter{remercier Goeffroy}
\todoRediger{Remercier Fongecif}
\todoRediger{Remercier Ingeus}
\todoRediger{Remercier IDApps, Yann, Stéphane, et surtout Laila!}

Enfin, last but not least, je tiens à remercier mon épouse Céline et mes deux filles pour leur encouragments, leurs soutient et leur patience.
\todoCompleter{Céline et les filles}

\todoCompleter{Completer les remerciements}
\todoMiseEnForme{Aérer}


\setcounter{tocdepth}{5}
\tableofcontents
\newpage

\chapter{Introduction}
\todoAmeliorer{L'introduction doit-elle être un chapitre ou alors un Abstract?}
\todoRediger{Introduction du context projet: j'ai effectué le stage blabla au sein de l'entreprise bla etc}
	\section{Tellus Environment - Missions Principales}
	\todoRediger{Tellus création, mission, activité}
	
	\subsection{Géophysique et cartographie haute-définition}
	Tellus Environmenent est une startup spécialisée dans la cartographie haute définition des sous-sols et des fond-marins. Elle propose à ses clients une offre bout en bout d'acquisition et de traitement des données pour permettre à ses clients d'agir en fonction de ces conclusions.
	\todoAmeliorer{Ce \S $\;$peut servir à l'intro sur Tellus.}
	
	\subsection{Collecte des données géophysique}
	
	\paragraph*{} Tellus Environment met en oeuvre ses propres équipements géoradar et magnetomètre pour l'acquisition de données sur le sous-sol sur des surfaces de quelques mètres-carré à quelque hectares. Elle est aussi en mesure de planifier et coordonner la mise en oeuvre d'équipements plus lourds - LIDAR, géoradar, magnétomètres aéroporté - pour obtenir des données sur des surfaces beaucoup plus importantes, de l'ordre du kilomètre carré.
	
	\missingfigure{Georadar}
	\missingfigure{Magnetomètre}
	\missingfigure{Radargramme}
	
	\paragraph*{} Tellus Environment peut aussi coordonner la mise en oeuvre d'équipements d'acquisition marins - sonar, multibeam, \underline{percuteur} \todoCheck{est-ce le bon terme?} - pour permettre la cartographie des fonds marins et des sous-sols aquatiques, que ce soit en environnement eau douce - rivières, lacs étangs - ou marins. 
	\missingfigure{sonar}
	\missingfigure{sonargramme}
	
	\paragraph*{} Tellus Environment a accès à de nombreuses bases cartographiques pour compléter les données acquises sous sa supervision pour aider à la mise en oeuvre des équipements ainsi qu'à compléter les données acquises en vue de leur traitement.

	\subsection{Traitement des données}
	\todoRediger{Expertise Tellus Traitement des données}
%	\todoObjectif{Montrer la pertinance de l'expertise de Tellus pour la cartographie 3D}

	\section{Activité R\&D}
	\todoRediger{Activité R\&D}
	
	Activité: R\&D: développer des composants qui mettent en oeuvre les expertises de TellusEnvironment pour créer des produits innovants, dont les composants seraient de plus réutilisables pour améliorer la productivité des services Géophysique.


	\section{Projet Symeter V2}
	\todoRediger{Context Projet}
		\subsection{Symeter V1}
		Symeter V1: rapide rappel
		\todoRediger{SymeterV1}
		
		\subsection{Symeter V2: Objectifs}
		\todoRediger{Symeter V2: objectifs}
		Symeter V2: Objectifs
		Symeter V2: Présentation du plan de projet
- test du simulateur Gazebo pour évaluer son utilité dans le projet Symeter2
- Simulation couverture lidar
- Montage des outils nécessaires au développement simulé du projet symeter
- Mise en place de la localisation: installation et tests
- Mise en place de l'acquisition des relevés.

\chapter{Technologies, Contraintes et Plan de Projet}
	Ce chapitre présente dans leurs grandes lignes les entrants et les contraintes mis en jeu par le projet Symeter V2. On y énumère notament les différents éléments à mettre en oeuvre pour monter le système Symeter V2: les capteurs l'environnement d'exploitation logicielle, les outils mathématiques et procédés à coordonner pour monter le prototype de Symeter 2.
	
	\section{Capteurs}
		Les capteurs à mettre en oeuvre seront un ou plusieurs LIDAR, un ou plusieurs IMU et un ou plusieurs GPS. Cette section décrit les fonctionnalités de ces capteurs de ces capteurs.
			
		\subsection{LIDAR}
		Un LIDAR est un équipement qui permet de prendre de nombreuses mesures de l'environnement à partir d'un laser. En exploitant les informations d'angle de pointage et en observant les caractéristiques du signal réfléchin tant du point de vu temporel (temps de vol, phase) que du point de vue de la forme du signal (puissance réfléchie, déphasage, distorsion, effet Doppler , ...), il est possible de mesurer de nombreuses caractéristiques de l'environnement.
		
		\paragraph*{} Le LIDAR est utilisé par Symeter V2 pour détecter la forme précise du chantier d'ensilage.
		
		\subsubsection{Détecteur Actif:}
		Le LIDAR est essentiellement un détecteur actif dont de le principe est identique à celui du radar: un pulse electromagnétique est émis dans une direction privilégiée par une partie émettrice, constituant un signal incident. 
		
		\para Ce signal incident est réfléchi par un object de l'environnement, générant un signal réfléchi qui retourne en direction du LIDAR. Ce signal réfléchi est reçu par un détecteur. 
		
		\para En comparant le signal émis et le signal reçu nous pouvons en déduire certaine caractéristiques de l'environnement.
		
		\paragraph*{Différents types de mesure: } Il existe deux grands types de LIDAR: \begin{description}
			\item[LIDAR  à "temps de vol":] les lidars de ce type déterminent la distance de la cible en mesurant le temps écoulé par entre le moment de l'émission du pulse et la réception du signal réfléchi par la cible.
			\item[LIDAR  "full waveform":] les LIDAR de ce type émettent un signal périodique continu et mesurent la distance de l'object détecté essentiellement par mesure de déphasage du signal réfléchi par rapport au signal émis. Les modifications de la forme du signal réfléchi et de son intensité permettent d'extraire d'autre informations à propos de la cible. 
		\end{description} 
		
		\subsubsection{Scan de l'environnement}
		En faisant varier la direction du bloc émetteur/récepteur du LIDAR, il est possible de faire plusieurs mesures de l'environnement. Le LIDAR mettant en oeuvre des procédés optiques, il est possbile de faire varier très rapidement son pointage et ainsi d'obtenir de très nombreux relevés de l'environnement.
		
		\paragraph*{Deux grands modes de scan de l'environnement:} Il existe deux grands modes de scan de l'environnement pour un LIDAR
		
		\begin{description}
			\item[Mode 2D ou  "planar":] Dans ce mode le LIDAR mesure l'environnement dans un plan unique, généralement en faisant tourner le faiseau de manière circulaire autour d'un axe.
			\item[Mode 3D:] Dans ce mode le LIDAR mesure des portions d'espace, généralement en combinant deux rotations simulatées du faiseau, soit en faisant tourner sur un même axe de révolution plusieurs faiseaux.
		\end{description}
	
	\missingfigure{mode 2D vs mode 3D}
		
		\subsubsection{LIDAR Hokuyo UTM-30LX-EW}
		Le projet Symeter V2 met en oeuvre un LIDAR de type Hokuyo UTM-30-LX-EW, le même que celui utilisépar Symeter V1. C'est un équipement robuste et précis qui est de plus bien connu par Tellus Environment.
		
		\para Cet équipement est basé sur un laser infrarouge (longueur d'onde de 905nm) pour scanner un champ semi-circulaire de 270$\degres$. C'est un LIDAR de type "planar, temps de vol".
		
		\para Il mesure la distances des objets à sa portée par pas de 0,25$\degres$. La distance de détection maximale est de 30m.
		\newline
		
		\begin{tabular}{|c|c|}
			\hline 
			Angle de Balayage & 270$\degres$ \\ 
			\hline 
			Résolution Angulaire & env. 0,25$\degres$ \\ 
			\hline
			Temps de Balayage & 25ms/balayage (40 balayages par seconde) \\
			\hline
			Distance de détection & Portée garantie: 0,1 à 30m \\
			\hline
			Résolution de la mesure & 1mm \\
			& $\sigma$ \\
			\hline
		\end{tabular} 
	
		\todoRediger{Décrire les caratéristiques exactes }
		\missingfigure{Photo du Hukyo}
		
		\subsection{IMU}
		Un IMU - Inertial Mesurement Unit - est un equipement comportant plusieurs capteurs inertiels, d'accélération et d'angles qui permettent avec un traitement des données approprié d'établir de suivre la pose du véhicule sur lequel il est monté.
		
		\subsubsection{Les grandeurs mesurées}
		Un IMU comporte généralement 3 accéléromètres linéaires et un gyroscope, et souvent un magnétometre.
		
		\para Le \textbf{gyroscope} permet de mesurer, avec précision et en continu, l'inclinaison du véhicule en terme de \textbf{roulis, tanguage et lacet}. 
		
		\para Les \textbf{accéléromètres} mesurent en continu les accélérations linéaires de la partie du véhicule sur lequel l'IMU est fixé, dans trois directions différentes. Ces directions sont généralement dénotées $x,y,z$, par analogie avec le repère cartésien. Simplement, les accélérations sont mesurées dans le repère du véhicule, mobile et variable et non dans un repère absolu immobile.
		
		\para En combinant les variations d'inclinaison et les mesures d'accélérations linéaires dans le repère mobile du véhicule, il est possible de reconstruire avec une certaine certitude la trajectoire du véhicule dans le repère absolu, en utilisant les équations de dynamique Newtonnienne.
		
		\missingfigure{Trajectoire et mesures}
		
		\para Comme indiqué ci-dessus, un IMU possède souvent un \textbf{magnétomètre}. Celui-ci permet de mesurer la direction et l'intensité du champ magnétique terrestre par rapport à l'IMU et permet d'en dériver le cap, la direction vers lequel le véhicule pointe sur la surface de la Terre. 
		
		\para Ce permet, en démarrant l'IMU alors que le véhicule est parfaitement immobile, d'obtenir les conditions initiales du vecteur d'état du véhicule, par rapport au repère odométrique.
		
		\para Par ailleur, reconstituer une position à partir d'informations d'accélération implique mathématiquement de procéder à une double intégration numérique. Ce procédé est donc sujet à une dérive dans le temps et doit donc être recalé par une mesure de pose issue d'un autre capteur.
		
		\para Par contre, un IMU peut fournir ses informations avec un très haut taux de rafraichissement, sans nécessiter d'information extérieure au véhicule. Il peut donc se révéler indispensable lorsqu'aucune autre nouvelle information sur la position du véhicule n'est disponible sur des périodes de temps plus ou moins longue.
		
		\subsubsection{IMU XSense}
		\todoRediger{données générées par un IMU}
		\missingfigure{graphes temporels des données générées par l'IMU}
		\missingfigure{Photo de l'IMU xsense}
		
		\subsection{GPS en mode RTK}
		
		Le système Symeter V2 ayant besoin d'une précision centimétrique pour permettre la mesure du tas d'ensilage avec précision du même ordre de grandeur, il mettra en oeuvre d'un GPS en mode RTK. Ce mode fonctionnement particulier du système GPS est en mesure de fournir position centimétrique du véhicule relativement à une borne GPS (base) situé à proximité du chantier.
		
		\subsubsection{GPS simple}
		Pour expliquer le principe du GPS en mode RTK, il est nécessaire de comprendre comment le système GPS "simple" fonctionne. 
		
		\para Le système GPS est un système de positionnement par satellite qui, en se basant sur les signaux émis par une constellation de satellites, est en mesure de calculer la position sur la surface de la Terre d'un récepteur avec une précision de l'ordre de la dizaine de mètres, pour une utilisation courante.
		
		\para Le principe de base est celui de la triangulation: les satellites GPS émettent en permanence un signal comportant une horloge (temps GPS). En connaissant les éphémérides des satellites et en écoutant les signaux de plusieurs de ces satellites, notamment leur signal d'horloge respectif, un récepteur GPS peut calculer avec une grande précision la position de chacun des satellites émetteurs dans le référentiel géocentrique.
		
		\para Une fois le positionnement des satellites établi, une simple triangulation permet de calculer avec une grande précision du récepteur sur la surface de la Terre.
		
		\para Pour obtenir une précision suffisante, un récepteur doit écouter au moins 4 satellites en même temps.
		
		\missingfigure{GPS en mode simple}
		
		\subsubsection{Principe du mode RTK}
		Le mode RTK du système GPS est un mode de fonctionnement qui permet de mesurer avec une très grande précision la position relative de deux récepteurs GPS distincts. L'un, dénommé \textbf{la base} est un récepteur fixe, dont la position GPS est connue avec une grande précision. L'autre dénommé \textbf{le rover} est en communication constante avec la base au moyen d'un liaison radio adaptée.
		
		\para Les deux récepteurs étant présents sur le même site à la surface de la Terre, ils seront à l'écoute tous les deux mêmes satellites. En comparant les signaux reçus d'un même satellite par la base et le rover, le rover est en mesure de calculer avec une très grande précision - de l'ordre du centimetre - la différence de distante entre la base et le satellite d'une part, et le rover et le satellite de l'autre part. 
		
		\missingfigure{GPS en mode RTK}
		
		\para En combinant ces différences de distance à partir de plusieurs satellites, le rover est donc capable de calculer très précisément sa position relativement à la base.
	
		\todoTrouverRefExt{GPS en mode RTK}
		
		
		\missingfigure{Photo base + rover RTK}
		
		
		
	\section{Environnement de programmation: ROS}
	Du fait de son utilisation dans la première version de Symeter, et du fait de ses qualités en terme de modularité, de disponibilité des drivers pour les capteurs et actionneurs, l'environnment logiciel ROS a été choisi avant même le début du stage pour être la base de la partie logicielle de Symeter V2.
	
	\missingfigure{Quelques robots powered by ROS}
	
	\para ROS est un environnement logiciel destiné à la mise en oeuvre de plateforme robotique. Cet environnement très modulaire permet de programmer et de déployer de nombreux modules interdépendants, appelés \textbf{packages}, constitués eux de noeuds exécutables appelés \textbf{nodes}.
	
	\para Ces nodes peuvent communiquer entre eux aux moyens de deux mécanismes principaux: \begin{itemize}
		\item les services
		\item les topics
	\end{itemize}

	\para Ces deux mécanismes permettent l'échange de données entre nodes selon des messages de la structure peut être spécifiée par le développeur du système.
	
	\para Les \textbf{services} sont mis à disposition chacun par un node. Ils peuvent être invoqués par d'autre node en effectuant une transaction explicite: le node appelant fourni une donnée d'entrée au node mettant à disposition le service. Ce dernier traite la donnée entrante puis génére une donnée de sortie qui est enfin renvoyée au node, terminant ainsi la transaction.
	
	\para Les \textbf{topics} sont des canaux de diffusion d'information similaire à une autoroute: tout node peut diffuser un message vers un topic et tout node peut écouter un topic pour recevoir et traiter les messages qui y sont émis.
	
	\missingfigure{Exemple d'une structure en packages, noden topics et services}
	
	\para Ces topics pourvoient des données que l'on pourrait qualifier de "sensorielle", dont la durée de vie est courte, telle que les données générées par un capteur fournissant des données en continu. 
	
	\para Ils sont aussi le support de diffusion des \textbf{transformations}, qui permettent de décrire l'état de \textbf{pose} de chacuns des composants mécaniques d'un véhicule autonome, ainsi que la pose du véhicule dans son environnement.
	
	\para ROS est un environnement fourni sous une licence Open Source et propose des drivers pour de très nombreux capteurs et actionneurs de tous types. Il intègre aussi des composants logiciels tiers spécialisés tels que, entre autre, la Point Cloud Library pour le traitement des nuages de points et la librairie OpenCV pour le traitement de la vision par ordinateur.
	
		
	\section{Les outils mathématiques}
	\todoObjectif{Introduire les concepts, indiquer qu'ils seront approfondis dans la suite du document.}
	
	\para Comme indiqué en introduction, le système Symeter V2 sera une plateforme semi-autonome, qui devra être en mesure de fournir une aide à la décision à partir de mesures effectuées depuis une plateforme mobile, et ceux avec une intervention humaine minimale.
	
	\para Symeter V2 devra donc estimer en temps réel sa pose, afin de pouvoir intégrer une représentation 3D du chantier afin de pouvoir effectuer son service.
	
	\para Les outils mathématiques qui devront être mis en oeuvres sont essentiellement ceux mis en oeuvre dans les systèmes robotiques:
	 \begin{itemize}
	 	\item Les \textbf{poses}, c'est à dire la représentation de la position des capteurs et des éléments mécaniques du véhicule les uns par rapport aux autres, et celle du véhicule par rapport à son environnement.
	 	\item L'estimation de paramètres à partir d'un ensemble de mesures issues de différents capteurs, au moyen de la \textbf{fusion de données}
	 	\item Reconstruction d'une scène 3D à partir de nuages de points
	 \end{itemize} 
 
   \para Ces outils sont décrits plus en détail dans les section suivants.
	
	
		\subsection{Positionnement en Robotique: Poses}
		
		La robotique à pour enjeu de permettre la mise en oeuvre de systèmes mécaniques autonomes, mobiles ou non, qui mettent en oeuvre une série de capteurs et d'actionneurs pour agir sur leur environnement avec une interaction humaine très limitées.
		
		\para La capacité de représenter de manière fine la position des différents éléments d'un tel système les uns par rapport aux autres, que ce soit en position $x,y,z$, mais aussi en inclinaison - $\theta_x$ (tanguage), $\theta_y$ (roulis), $\theta_z$ (lacet) - est donc primordiale. L'état de l'élément décrit par ce vecteur $x,y,z, \theta_x, \theta_y, \theta_z$ est appelé sa \textbf{pose}.
		
		\missingfigure{Pose 3D d'un objet}
				
		\para Les outils mathématiques de prédilection pour décrire une pose sont bien sur les transformations dans $\mathbb{R}^3$, translations et rotations. Cependant le système de coordonnées classique à 3 dimensions spatiales $x,y,z$ ne permet pas de combiner simplement les transformation 3D, ce qui est un besoin de base pour passer d'un réferentiel à un autre.
		
		\para C'est pour cela que la robotique fait une utilisation intensive des \textbf{Coordonnées Homogènes}, issues de la géométrie projective, qui permettent de combiner les transformation 3D en passant par un espace à 4 dimensions approprié. Les transformations y sont représentées par des matrices de $\mathbb{R}^{4 \times 4}$, ce qui permet de combiner les transformations par simple multiplication de matrices.
		
		\para Les principes de base des coordonnées homogènes sont décrites en Annexe XXX \todoAjouterRef{Lien vers l'annexe}
		
		\para Les rotations dans l'espace sont aussi représentables par des quaternions de norme unitée. Cette représentation est souvent préférées par la robotique par rapport à une représentation basée sur les rotations eulériennes car elle permet de combiner les rotations sans rentrer dans un "gimbal lock". 
		
		\para ROS utilise les quaternions de manière native dans la description de ses transformations. \todoAjouterRef{Quaternion et quaternions unité}
		
			
		\subsection{Localisation par fusion de données}
		\label{intro-loc-fusion-donnees}
		
		Afin de pouvoir reconstituer le chantier en une représentation 3D avec une précision centimétrique, il est nécessaire que pour chaque mesure LIDAR le système Symeter V2 aie une estimation suffisamment précise de la pose du véhicule, afin de déterminer la pose du LIDAR et pouvoir situer dans le chantier la position $x,y,z$ de chaque point mesuré.
		
		\para Le problème de la localisation du véhicule est donc le suivant:
		\para Estimer à chaque instant $t_n$ le vecteur d'état:
		\begin{equation}
		\mathbf{X} = \begin{bmatrix}
			x_n \\ y \\ z \\ \theta_x \\ \theta_y \\ \theta_z
		\end{bmatrix}
		\end{equation}
		du véhicule à partir des données disponibles.
		
		\missingfigure{Flux de données IMU + GPS --> $X$}
		
		\para Les sources de données disponibles pour estimer la pose du véhicule à chaque instant sont:
		\begin{itemize}
			\item L'IMU,
			\item Le GPS en mode RTK.
		\end{itemize}
		
		\para L'IMU fourni des informations sur l'inclinaison du véhicule, ainsi que les accélérations linéaires qu'il subit à une fréquence de l'ordre de 30 Hz. Cependant comme indiqué ci-dessus, calculer la position du véhicule à partir de la mesure des accélérations subies requiert une double intégration, sur un signal d'accélération qui estgénéralement très biruité. Ces éléments impliquent une forte probabilité de dérive dans le temps.
		
		\para Le GPS en mode RTK peut fournir une mesure de la position du véhicule dans la scène avec une précision de l'ordre de quelques centimètres, mais à un taux de rafraichissement beaucoup plus bas, de l'ordre de 2 à 5 Hz. De plus le GPS ne fournit aucune information sur l'assiette du véhicule.
		
			
		\todoRediger{introduire les filtres de Kalman}
		\missingfigure{chaine de traitement de fusion de données}
		\todoAjouterRef{Filter de Kalman pour la fusion de données.}
		\todoCompleter{Comme le GPS fourni une position absolue dans le référentiel de la scène nous ne sommes pas dans une problématique SLAM. Le véhicule n'est pas autonome, la navigation est assurée par l'opérateur du tracteur.}
			
		
		\subsection{Traitement des nuages de points}
		Le LIDAR utilisé par le système Symeter V2 étant de type "2D planar", les points retournés lors d'une unique mesure feront tous partie du même plan. L'ensemble des mesures sera donc constitué d'un ensemble de tranches qu'il faudra reconstituer à l'aide d'un traitement adapté.
		
		\para Le traitement des données LIDAR comportera les étapes suivantes:
		\begin{enumerate}
			\item Filtrage des données pour une obtenir une densité de points de mesure qui soient suffisament espacés
			\item Convertion des données LIDAR (direction / distance) en un nuage de points (liste de point $P_n$ avec chacun des coordonnées $x_n, y_n, z_n$
			\item Stockage de ces points dans une structure de données capable d'accumuler les nouveaux points mesurés au cours du temps, et constituer une carte 3D d'occupation.
 		\end{enumerate}
 	
 		\para La plupart de ces traitements seront implémentés en utilisant le Point Cloud Library qui offre de nombreux composants logiciels intégrés directement dans ROS.

		\missingfigure{Architecture de traitement des nuages de points}
			
	\section{Contraintes de développement}
		Le système Symeter V2 st destiné à évoluer sur un véhicule de type tracteur agricole. Il est de plus destiné à construire une représentation 3D d'un chantier d'ensilage, partir de positions qui vont varier non seulement en $x$ et en $y$, mais aussi et surtout en $z$.
		
		\subsection{Capacités de tests en grandeur limitées}
		Pour développer et tester le système Symeter V2, le développeur ne disposera pas de tracteur en grandeur réelle, ni de chantier d'ensilage, ceux-ci étant des chantiers annuels se déroulant à des moments bien défini dans l'année (printemps et automne).
		
		
		\para Par ailleurs, au début du stage il n'est pas encore tout à fait décidé de combien de LIDAR seront nécessaires pour que le système puisse assurer une couverture complète depuis des équipements embarqués sur le tracteur.
		
		\para Les équipements à mettre en oeuvre sont relativement couteux et leur mise en oeuvre requiert une certaine expertise. Il est donc nécessaire de pouvoir mettre en oeuvre le développement du système par le biais de mises en oeuvre alternatives.
		
		\para Ces possibilités sont au nombre de 2: la simulation et la mise en oeuvre en grandeur des équipements sur une camionnette pour effectuer des tests simples de reconstitution du terrain.
		
		\subsection{Plateformes de test disponibles}
			
			\subsubsection{Environnement de simulation Gazebo}
			
			Gazebo est un environnement de simulation très complet, qui permet de simuler un environnement physique dans lequel une plateforme semi-robotique simulée peut évoluer. 
		
			\todoRediger{De nombreux capteurs virtuels disponibles, possibilité de faire varier les incertitudes}
			
			\para Gazebo est un logiciel intégré à ROS. Sa mise en oeuvre requiert de programmer un véhicule virtuel similaire à la plateforme cible du système (dans notre cas un tracteur avec les capteurs listés dans les sections précédentes), puis de déployer le logiciel ROS pour Symeter V2 sur ce véhicule simulé. 
			
			\para Il est aussi possible de spécifier le "monde" dans lequel le tracteur simulé évoluera, en y ajoutant des murs, du relief, voire une butte en guise de tas d'ensilage.
		
			\missingfigure{Capture d'écran Gazebo avec le tracteur et le silo}
			
			\para Gazebo est donc utilisé pour simuler un chantier d'ensilage pour tester la localisation, l'acquisition du terrain, la mesure d'un modèle de Silo.
			
			\subsubsection{Prototype monté sur Camionnette}
			
			Une simulation n'étant jamais parfaite (comme nous le verrons par la suite), il est quand même nécessaire d'effectuer des certains tests avec des instruments réels, lors de tests en grandeur.
		
			\missingfigure{Maquette avec équipements réels montés sur une camionnette.}
	
				
			\para Pour la mise en oeuvre en grandeur, un prototype de l'équipement a été monté sur la camionnette de Tellus Environnement, la "Tellus Car".
			
			\para Le prototype comporte:
			\begin{itemize}
				\item Une unité de calcul
				\item Un GPS en mode RTK
				\item Un LIDAR de type Hkuyo UTM-30LX-EW
				\item Un IMU de type XSense
			\end{itemize}
		
			\para Ce prototype a été utilisé pour effectuer des captures des trois capteur alors que la Tellus Car évoluait dans le parking adjacent aux locaux de Tellus Environnement.
		
	\section{Les grandes phases du stage}
	
	Le passage de Symeter V1, où le LIDAR est fixe par rapport au chantier, à Symeter V2 où le LIDAR est monté sur le tracteur et est donc mobile par rapport au chantier, remet tout en question vis-à-vis du travail effectué sur Symeter V1.
	
	\para Le fait d'embarquer le système sur un tracteur pose une contrainte très forte sur le projet en forçant l'ajout d'IMU et d'un GPS qui devront être utilisé pour la localisation.
	
	\para Donc d'un sujet V1 où nous avions une reconstition du chantier par une accumulation simple de scans LIDAR rectangulaires, nous avons en V2 deux  problèmes complexes interdépendants:
	\begin{itemize}
		\item Localisation à partir de fusion de données IMU et GPS
		\item Reconstition du chantier à partir de scans aléatoires
	\end{itemize}

	\para Un problème supplémentaire étant l'absence d'un plateforme matérielle de test pour pourvoir tester les solutions de ces deux problèmes.
	
	\para Comme il était apparent que sur tous ces sujets nous partions de zéros, il a été décidé très tôt dans le projet de monter une plateforme de tracteur simulée sous Gazebo pour effectuer le développement initial de Symeter V2. Cette plateforme simulée nous permettrait dans un premier temps d'effectuer des tests qualitatifs des solutions choisies, pour valider un certain niveau de faisabilité ainsi que faire des essais sur le choix du nombre de capteur et leur disposition.
	
	\para A plus long terme, cette plateforme simulée nous permettra d'effectuer des tests quantitatifs de Symeter V2 lorsque de l'implémentation de le mesure de tassage sera complétée, et donc de valider le bon fonctionnement du système avant même de faire des tests sur des chantiers d'ensilage en grandeur.
	
	\para Le déroulement du stage a donc été le suivant:
	\begin{enumerate}
		\item Vérification des capacités de Gazebo à simuler notre plateforme (capteurs, véhicule)
		\item Montage de la plateforme Simulée tracteur et capteurs sous ROS / Gazebo
		\item Montage du processus de localisation par fusion IMU + GPS
		\item Montage de la chaine de traitement LIDAR pour reconstitution de la représentation 3D du chantier
		\item Tests des éléments développés ci-dessus sur des captures issues du prototype monté sur la Tellus Car.
	\end{enumerate}

	\para Les résultats de ces différentes parties du stage sont décrits dans les chapitres suivants de ce rapport.
	
				

\chapter{Simulation d'un tracteur évoluant sur un chantier d'ensilage à l'aide de ROS/Gazebo}
\label{chap-simu-ros-gazebo}

Comme indiqué dans le chapitre précédent, il a été très tôt dans le projet décidé d'utiliser l'environnement logiciel ROS pour implémenter le système Symeter V2. Il avait aussi été décidé de monter un véhicule simulé à l'aide de Gazebo afin d'effectuer le développement initial.

\para Ce chapitre présente en détail l'environnement logiciel ROS et le logiciel de simulation Gazebo. Cette présentation est illustrée par la construction d'un robot virtuel simple, pour lequel un retour d'expérience est effectué en fin de chapitre suite à cette première expérience.

\para L'implémentation du tracteur de test est ensuite présentée en fin de chapitre.

	\section{Présentation de ROS}
	
	ROS - Robot OS - est un environnement logiciel destiner au pilotage autonome de systèmes robotiques. Ce logiciel Open Source originellement développé conjointement par Willow Garage et l'université de Standford, a été lancé pour sa version 1.0 en janvier 2010 (voir www.ros.org).
	
	\para ROS en est maintenant à sa 11\textsuperscript{ème} distribution, dénommée "ROS Lunar Loggerhead", distribution qui sera utilisée pour Symeter V2.
	
	\missingfigure{Photo d'exemples de robots, roues, bras, humanoides, etc}
		\subsection{Architecture de ROS}
		\todoRediger{ROS est un environnement permettant le montage de plateformes robotiques complètes.}
		\todoRediger{Architecture logicielle très modulaire}
		\todoRediger{Nodes, services, topics, capteurs, etc}
		\missingfigure{Exemple de node}
		\subsection{Gestion des transformations}
		\todoRediger{précision d'évolution d'un plateforme robotique dépend de la prise en compte de la position relative de ses différents capteurs et actionneurs}
		\subsection{Gestions des Capteurs}
		
		\todoRediger{un petit laïus surt les capteurs supportés}
	\section{Présentation de Gazebo}
	
	Gazebo est un logiciel qui permet de simuler un "monde" virtuel 3D, dans lequel il est possible de plonger des véhicules robotiques qui seront alors soumis aux lois physiques simulées.
	
	\missingfigure{Capture d'écran de gazebo}
		\subsection{Construction d'un robot virtuel}
		
		Un robot virtuel sous Gazebo est décrit par un fichier XML au format SDF. Ce format permet de spécifier un robot en ses éléments constitutifs simplifiés (link) chassis, roue, partie de bras, etc) de décrire comment ces éléments sont placés les uns par rapport au autres.
		
		\para Chaque élément  du robot doit comporter 3 parties différentes
		
		\para Ce robot doit faire partie d'un monde qui est lui même décrit par un fichier SDF. 
		
		\todoTrouverRefExt{SDF}

		\todoRediger{URDF et SDF, world, contrôleurs}
		
		
		\subsection{Vérification de disponibilité des capteurs}
		
		Gazebo propose tout une série de capteurs simulés, disponible nativement ou par l'intermédiaire de module tiers. Ces capteurs simulés génére une donnée en cohérence avec l'environnement simulé qui peut être rendu disponible aux modules ROS par l'intermédiaire de plugins adaptés.
		
		\para Pour les besoins du projet Symeter V2, nous avons vérifié et testé la disponibilité des types de capteurs tels que listés dans le tableau suivant.
		\newline
		
		\begin{tabular}{|c|c|c|}
			\hline 
			Capteur & Plugin Gazebo & Utilisabilité \\ 
			\hline 
			IMU &  &  \\ 
			\hline 
			LIDAR &  &  \\ 
			\hline 
			GPS &  &  \\ 
			\hline 
		\end{tabular} 
	
		\para Pour vérifier la disponibilité et le bon fonctionnement de ces capteurs, nous avons utilisé un petit robot basé sur une implémentation pré-existante décrite dans \todoAjouterRef{le tuto de petit robot}.
		
		\missingfigure{Le petit robot}
		

		
\section{Contraintes de mise en oeuvre}
	
	\para La mise en oeuvre de ce petit robot au sein de Gazebo a été riche d'enseignements. Elle a permis notamment de détecter quelques problèmes génant dans l'utilisation de Gazebo.
	
		\subsection{Pas d'adhérence au démarrage de la simulation}
		
		La mise en oeuvre du robot de base a révélé rapidement révélé que Gazebo comporte quelques problèmes  lié à la simulation de l'adhérence des roues avec les objets environnants, notamment le sol.
		
		\para En particulier, les roues d'un robot ROS plongé dans Gazebo n'auront aucune adhérence sur le sol au démarrage de la simulation. Si la propulsion est mise en oeuvre, nous pouvons constater que les roues tournent mais que le robot ne bouge pas, comme si les roues tournaient dans le vide, ou sur de la glace.
		
		\para Après avoir changé des réglages dans le moteur physique de Gazebo, dans le réglage du coefficient de frottement des roues et de nombreux autres paramètres sans effet sur le problème, un contournement a été mis en place pour le mitiger: un node ROS a été écrit pour forcer la position du robot à des coordonnées spatiale bien précises, en utilisant le topic \verb|/gazebo/set_model_state|.
		
		\para Après ce forçage, Gazebo prend ensuite bien en compte l'adhérence des roues du robot et le robot bouge quand les roues tournent. 
		
		\para Cela fait que ce script, dénommé \verb|setpose|, doit être invoqué systématiquement après le démarrage de la simulation pour que l'adhérence des roues soit bien prise en compte par le moteur physique de Gazebo.
		
		\subsection{Simulation mécanique, frottements, adhérence}
		
		Simuler un robot avec des points de contact multiples dans Gazebo peut poser parfois problème si les points multiples dérapent. Dans ce cas il est rare que L'adhérence soit récupérée.
		
		\para Ceci est exacerbé par le fait que les matériaux simulés par Gazebo sont par défaut infiniment rigides et que le moindre choc se répercute dans les points de contact avec le sol. 

		
		\subsection{Conclusions sur les contraintes}
		\label{conc-contrainte-gazebo}
		Pour conclure avec les limitations et contraintes de Gazebo, la leçon principale est qu'il faut au maximul éviter que le véhicule entre en dérapage: il ne récupère généralement pas, et devient inutilisable: il faut relancer la simulation.
		
		\para Les raisons pour lesquelles le véhicule peut entrer en dérapage sont nombreuse:
		\begin{itemize}
			\item Mauvaise configuration des paramètres de frottement des éléments en contact avec le sol.
			\item  Vitesses de rotation différentes des roues, soit par une mauvaise consigne de vitesse sur l'une des roues, soit une non prise en compte du différentiel de vitesse entre roues intérieur et extérieur dans un virage, etc
			
			\item En virage, les roues intérieures et extérieures sont soumises à des rayons différents. S'ils ne sont pas pris en compte correctement, une ou plusieurs roues sont succeptible de déraper.
		\end{itemize}

		
	\section{Mise en Oeuvre: simulation d'un environnement de tassage de silo}
		\subsection{Montage d'un tracteur simulé}
		Intérêt: simuler l'implantation physique des capteurs avec des dimensions du même ordre de grandeur que les plateformes cibles.
		
		\subsubsection{Description Physique}
			\paragraph{Chassis}
			Le chassis du tracteur à été monté en se basant sur les dimensions générales d'un tracteur de type \textbf{TODO: ICI TYPE DE TRACTEUR} \todo{Ecrire type de tracteur}, en utilisant cependant des formes géométriques simplifiées.
			\newline
			
			Le tracteur est donc composé d'un pavé en guise de chassis, de 4 cylindres allongés en guise d'essieux, et de 4 grands cylindres en guise de roues.
			
			 \missingfigure{Image du tracteur modélisé}
			\paragraph{Actuateurs et Contrôleurs}
			
			Pour permettre une conduite en terrain accidenté, le tracteur simulé sera muni de 4 roues motrices, avec 2 roues directrices à l'avant.
			
			\para Chacune des roues motrices est commandée en vitesse par son propre topic, telle que décrit dans le tableau suivant, et l'angle de chacune des roues de direction est commandée par un topic dédié.
			
			\missingfigure{Tableau avec la correspondance entre vitesse roue et topic, angle direction et roue directionnelle}
		\subsubsection{Propulsion et Guidage}
		
		
		
			Du fait des contraintes exposées dans le paragraphe \ref{conc-contrainte-gazebo}, la consigne de vitesse sur chacune des roues doit être cohérente vis-à-vis de la vitesse de consigne, le rayon de chacune des roues ainsi que le rayon de virage imposé.
			
			\paragraph{Algorithme}
			\label{algo-steering}
			
			\todo[inline]{Description géométrique du problème de direction différentielle}
			
			\paragraph{Implémentation sous ROS}
			
			L'implémentation sous ROS de cet algorithme est effectué au moyen d'un node dédié, nommé \verb|tracteur_steering.py|, qui fait partie du package \verb|tracteur_control|.
			
			\para Le node \verb|tracteur_steering| prend en entrée le topic \verb|/tracteur/cmd_vel| qui comporte des messages de type \verb|Twist|. Ce message contient deux informations: l'un comporte la consigne de vitesse du tracteur, et l'autre le taux de rotation à gauche ou à droite du tracteur.
			
			\para Le node \verb|tracteur_steering| calcule sur la base des consignes données 6 nouveau paramètres, sur la base des algorithmes exposés en \ref{algo-steering}:
			\begin{itemize}
				\item les 4 consignes de vitesse pour chacune des roues
				\item les 2 consignes de direction pour chacune des roues directionnelle.
			\end{itemize}
		
			\para Ces consignes sont soumises à chacun des contrôleur par l'intermediaire des topics qui leur sont dédiés \todoAjouterRef{tableau avec les topics}.
			
			
			
			\todo{Description de l'implémentation sous ROS}
			\missingfigure{Schéma des flux topic qui permet de transformer une commande twist en 4 commandes de velocités.}
		
		
		

\chapter{Mise en place du processus de localisation}

Ce chapitre présente la problématique de localisation par fusion de données IMU et GPS. Malgrè l'objectif de mettre en oeuvre des composants disponibles "sur étagère" pour monter la version initiale de Symeter V2, il est quand même nécessaire de comprendre en détail les tenants et les aboutissants du problème de la localisation.

\para Ce problème est complexe et requiert d'en comprendre au moins les bases afin de permettre de configurer les composants correctement.

	\section{Présentation du problème}
	Comme indiqué dans le paragraphe \ref{intro-loc-fusion-donnees}, le but du processus de localisation est d'estimer en temps réel pour chaque instant $t_k$ le vecteur d'état 
	
	\begin{equation}	
	\bs{x}_k = \begin{bmatrix}
	x_k \\ y_k \\ z_k \\ \theta_{x_k} \\ \theta_{y_k} \\ \theta_{z_k}
	\end{bmatrix}
	\end{equation}
	\newline
	
	sur la base d'informations fournies par l'IMU et le GPS en mode RTK, ce vecteur d'état étant exprimé dans un référentiel fixe par rapport au chantier.
	
	\missingfigure{Représentation du vecteur d'état par rapport au chantier}
	
	\para Selon \cite{gustavsson_uav_2015}, il est possible de mettre en place un procédé qui maintient ce vecteur d'état sur la base de ces deux instruments.
	
	\subsection{Repères}
	
	\para L'IMU et le GPS peuvent fournir des informations à certains instant $t_k$ qui permettront de mettre à jour le vecteur d'état pour l'incrément de temps suivant $t_{k+1}$. Cependant, les données exposées par les capteurs ne sont pas toutes dans le référentiel fixe par rapport au chantier.
	
	\para L'IMU par exemple expose des composantes d'accélération linéaire qui sont attachées au référentiel de L'IMU lui-même, supposé fixe par rapport au véhicule, mais qui est donc mobile par rapport au chantier.
	
	\para Le GPS lui n'est en mesure que de donner des mesures de positions que dans un référentiel lié à la surface de la Terre, en terme de lattitude, longitude et altitude $(\lambda, \phi, h)$ et il faudra donc transformer cette position géocentrique en coordonnées du référentiel de chantier.
	
	\para Pour travailler au problème de localisation, nous avons donc besion d'au moins trois référentiels différents pour maintenir le vecteur d'état $\bs{x}_k$ à partir des mesures de l'IMU et du GPS.
	
	\para Selon \cite{gustavsson_uav_2015}, nous pouvons donc distinguer les 4 référentiels suivants
	
	\subsubsection{Repère Inertiel}
	
	\subsubsection{Repère ECEF}
	
	\subsubsection{Repère de navigation local}
	
	\subsubsection{Repère du véhicule}
	
				
	\subsection{Système de Navigation Inertielle}
	
	\para Selon \cite{gustavsson_uav_2015}, les données de l'IMU peuvent être exploitée par un Système de Navigation Inertiel. Un tel 
	
	
	\subsection{Transformation des coordonnées géocentriques en coordonnées locale}
	

	
	\section{Filtres de Kalman}
	
	Selon \cite{gustavsson_uav_2015} et \cite{menegatti_generalized_2016}, a fusion de données IMU et GPS peut être effectuée sur la base d'un filtre de Kalman. Cette section s'attache a décrire les principes de fonctionnement d'un tel filtre pour en permettre l'utilisation pour le projet Symeter V2.
	
	Un filtre de Kalman est un algortihme qui permet d'estimer les états d'un système dynamique sur la base  de mesures incomplètes ou bruitée. Il a été développé à la fin des années 50 et comprend de très nombreuses applications dans tous les domaines liés à l'instrumentation.
	
	\subsection{Filtres de Kalman Linéaires}
	Selon \cite{zarchan_fundamentals_2009}, les filtres de Kalman sont applicables sur des systèmes dynamiques linéaires modélisés par un ensemble d'équations différentielles, décrites par la relation suivante:
	
	\begin{equation}
	\boldsymbol{
		\dot{x} = Fx + Gu + w
	}
	\end{equation}
	\todoCompleter{avec mes notes sur le filtre de Kalman Etendu.}
	
	\subsection{Filtres de Kalman Etendus}
	\todoRediger{pour les systèmes dynamiques non linéaires}
	
	\section{Mise en oeuvre sous ROS.} Comme les sections précédentes le montrent, l'implementation d'un système de localisation sur la base mesure par IMU et GPS ne sont pas une mince affaire et est plus un sujet de thèse qu'une sous partie de stage. 
	
	\para L'environnement ROS propose des modules prêts à l'emploi pour implémenter un processus de localisation pour les systèmes de robotique mobile. L'offre est cependant nombreuse et il a fallu vérifier que les solutions proposées répondent bien aux besoins de Symeter V2.
	
	\para Le besoin le plus important est que la localisation doit être effectuée sur un vecteur d'état compatible avec une scène 3D. Cela veut dire que le vecteur d'état estimé doit être d'au moins 6 dimensions: 3 pour la position spatiale du véhicule et 3 pour son assiette.
	
	\para Il a donc fallu éliminer tous les modules qui se limitaient à des évolutions 2D du système, d'autre qui n'acceptaient qu'un nombre limité de capteurs, ou alors qui ne supportaient pas le GPS, par exemple.
	
	\para Le choix final s'est finalement reposé sur le module \verb|robot_localization|, qui est décrit plus en détail dans les sections suivantes.
	
	\subsection{Module robot\_localisation}
	Selon le site internet de \verb|robot_localisation| (\href{http://docs.ros.org/melodic/api/robot_localization/html/index.html}{lien}), ce module est une collection de nodes dédiés à l'estimation d'état. Chacun de ces nodes implémente un estimateur d'état non linéaire pour des véhicules évoluant dans un espace en 3 dimensions.
	
	\para Il contient contient notamment le node \verb|ekf_localization_node| qui est une implémentation d'un filtre de Kalman Etendu, et le node \verb|navsat_transform_node| qui aide à l'intégration des données GPS.
	
	\para Les caractéristiques principales du node \verb|ekf_localization_node| sont les suivantes:
	\begin{itemize}
		\item Possibilité de fusionner un nombre arbitraire de capteurs
		\item Possibilité de prendre en compte de nombreux types de données différentes: odométrie, IMU, Pose avec covariance, etc.
		\item Possibilité de régler la configuration du node capteur par capteur
		\item Estimation en continu: dès que le node reçoit une mesure, il estime en continu le vecteur d'état, même en l'absence prolongée de nouvelle mesure, en utilisant un modèle interne pour la dynamique du véhicule.
	\end{itemize}

	\para Pour estimer en continu l'état du véhicule, le node \verb|ekf_localization_node| maintient en interne un vecteur d'état à 15 dimensions:
	
	\begin{gather}
		\begin{bmatrix}
			x&y&z&\theta_x&\theta_y&\theta_z&
			\dot{x}&\dot{y}&\dot{z}&\dot{\theta_x}&\dot{\theta_y}&\dot{\theta_z} &\ddot{x}&\ddot{y}&\ddot{z}
		\end{bmatrix}
	\end{gather}
	
	\subsection{navsat\_transformation\_node}
	\todoRediger{navsat transformation node}
	
	\subsection{Configuration des Capteurs}
	\missingfigure{Architecture des nodes}
	\todoRediger{Les variables d'état influencées par IMU}
	\todoRediger{les variables d'état influencées par le GPS}
	
	\subsection{Quelques tests}
	
	
\chapter{Exploitation des données LIDAR}

Ce chapitre présente la fonction acquisition et exploitation des données LIDAR pour le projet Symeter V2. 

\para Nous nous attachons à mettre en place les principes de traitement des données LIDAR en vue de construire une représentation 3D du chantier, d'abord en montant le pipeline de traitement du flux de données LIDAR, puis en mettant en oeuvre une structure de données adaptée au stockage d'une scène 3D.

	\section{Présentation de la chaine de traitement des données LIDAR}
	
	La chaîne de traitement des données LIDAR est représentée dans la figure \todoAjouterRef{figure} et comporte les éléments suivants:
	\begin{itemize}
		\item Une source de données de type \verb|laser|, issue des drivers ROS qui pilotent l'équipement LIDAR
		
		\item Un module de mise en forme du signal, qui prend en entrée les données \verb|laser|, les filtre de manière à uniformiser la repartition des points de mesure, à réduire la bande passante et transformer les données laser en un nuage de points.
		
		\item Un module d'accumulation de nuages de points, qui prend en entrée les nuages de points issus du module de mise en forme en vue de les aggréger et reconstruire la chaine.
	\end{itemize}
	
	\missingfigure{Chaine de traitement LIDAR --> Nuage de Point}
	\missingfigure{Transformer un ensemble de lignes en un nuage de points cohérent}
	
	\section{Acquisition et mise en forme des données LIDAR}
	L'acquisistion initiale des données LIDAR est effectuée par un driver approprié, généralement fourni par le constructeur de l'équipement. Les données sont structurée selon un format comportant la distance mesurée jusqu'au premier obstacle pour chaque incrément d'angle du LIDAR.

	
		\subsection{Transformation trame LIDAR en un nuage de points}
	
		\para La première étape du traitement est donc de transformer ces données de type \verb|laser| (au sens de l'environnement ROS) pour générer un nuage de points dans le référentiel du chantier.
	
		\todoRediger{LIDAR: ensemble d'angles, temps de vol, puissance reçue}
		\todoRediger{Nuage de points: positionnement x,y,z dans la scène}
	
		\subsection{Filtrage de la ligne de point par downsampling}
	
		\para A raison de 40 scans par secondes et de 1080 points mesurés par scan, un LIDAR génére 43200 points par secondes, ce qui est beaucoup.
	
		\para De plus, du fait que la hauteur du LIDAR monté sur un tracteur sera au maximum d'environ 3 à 4 mètres de hauteur, et que l'acquisition des mesures est effectué sur la base d'un incrément d'angles, la densité des points de mesure sera beaucoup plus hétérogène que pour Symeter V1. Cela veut dire que nous aurons beaucoup plus de points de mesure directement sous le LIDAR que sur les côtés.
	
		\para Nous allons donc effectuer un traitement qui permettra de moyenner les mesures de points autour de positions régulièrement réparties tous les 5 cm (par exemple). Cela permettra de réduire le nombre de points à incorporer dans la scène,  tout en régularisant la répartition des points dans l'espace.
		
		\para D'aprés \cite{moreno_comparative_2016}, il existe plusieurs méthodes de filtrage de nuages de points. Ceci est effectué en utilisant un procedé basé sur les Voxels
	
		\todoCompleter{principe de fonctionnement des Voxels} 

		\para Le résultat final de ce traitement est que d'un nuage de points irrégulièrement espacé et avec beaucoup de bande passante nous obtenons un nombre réduit de points régulièrement espacés, idéalement conditionné pour être pris en compte de manière fiable par l'accumulateur.
	

		\todoRediger{Il est nécessaire de connaitre précisement la pose du LIDAR pour générer le nuage de points.}
		
		\missingfigure{Illustration nature donnée LIDAR}
		\missingfigure{Illustration nature donnée Nuage de Point}
		\missingfigure{Diagramme fonctionnel LIDAR --> Nuage de Point}

		
	\section{Accumulation des nuages de points}
	
	Une fois les données LIDAR mises en forme, il convient maintenant de les exploiter pour reconstituer la représentation 3D du chantier. Cette tâche est effectuée par un module que l'on appelle le \verb|point_cloud_aggregator|. Comme son nom l'indique ce module aggrège et accumule les nuages de points issus de l'aquisition pour générer la représentation 3D du chantier.
	
		\subsection{Spécifications pour le point\_cloud\_aggregator}
		Le \verb|point_cloud_aggregator| dans un premier temps agrège les nuages de points, c'est à dire qu'il prend un nuage de point en entrée, et tente de les ajouter au nuages de points qu'il a déjà en mémoire. 
		
		\para S'il tente d'ajouter un point à une position dans l'espace pour laquelle un point existe déjà en mémoire (collision), ce nouveau point n'est pas ajouté. Cela évite ainsi les points doubles.
		
		\para Cela met cependant une contrainte forte sur la structure de données stockant les points déjà accumulés: il faut pouvoir identifier de manière unique tous les éléments de volume inclus dans la représentation de la scène afin de pouvoir détecter les collisions.
		
		\subsection{Principe de stockage des données 3D}
		\todoRediger{les structures de données de stockage de l'information 3D}
		\todoTrouverRefExt{les structures de données 3D}
		
		\subsubsection{Les B-Trees}
		\todoRediger{Les b-trees, principes}
		\todoRediger{Point forts: calculs performants}
		\todoRediger{points faible: arbre équilibrés --> difficile d'ajouter de nouveaux points.}
		
		\subsubsection{Les Octrees}
		\todoRediger{Octree, principes}
		\todoRediger{points forts: structure déséquilibrées sans problème, possibilité d'ajouter de nouveau points avec une bonne performance}
		\todoRediger{point faible: beaucoup d'overhead de memoire si pas implémenté correctement.}
		
		\subsubsection{Le choix: octree correspond à notre besoin.}
		
		\subsection{Mise en oeuvre: octomap}
		\todoRediger{Utilisation d'octomap car composant sur étagère}
		\todoRediger{permet d'implémenter rapidement la chaine de traitement pour vérifier la validité de la faisaibilité}
		
	\section{Mise en oeuvre sous Gazebo}
	
	La section précédente décrivait les principes de traitement des données LIDAR en vu de construire une représentation 3D d'un chantier d'ensilage. La présente section va illustrer la mise en oeuvre de ces principes à l'aide la plateforme de simluation décrite dans le chapitre \ref{chap-simu-ros-gazebo} évoluant dans un chantier d'ensilage simulé.
	
		\subsection{Modélisation d'un chantier d'ensilage}
		Le modèle simulé du chantier d'ensilage est représenté pour 2 murs chacun long de 10 mètres, haut de 3 mètres et d'épaisseur 0,2 mètre, espacés de 9 mètres. Ce modèle est créé en utilisant les primitives XML fournies par le logiciel Gazebo.
		
		\missingfigure{Capture d'écran représentant le mur sous Gazebo.}
		
		\para Un tas d'ensilage peut aussi être ajouté au chantier. Ce modèle de tas d'ensilage a été contruit en utilisant dans un premier temps le logiciel de modélisation 3D \verb|blender|. Ce modèle 3D est ensuite incorporé dans le chantier simulé en l'important dans la description du world 3D de manière appropriée.
		
		\missingfigure{Capture d'écran modèle d'ensilage sous blender + capture d'écran mur et ensilage dans Gazebo}
		
	
	
		\subsection{Test sous gazebo}
		Une fois la simulation de chantier montée, il est possible de faire évoluer le tracteur simulé dans le chantier. L'opérateur peut ainsi diriger le tracteur de manière à ce que le LIDAR monté à l'arrière du tracteur puisse scanner séquentiellement dans son ensemble le silo.
		
		\missingfigure{Capture Gazebo du chantier avec tracteur et mur}

		\todoRediger{Une fois l'acquisition effectuer, on invoque le node de sauvegarde de nuage, qui le sauve dans un fichier}
		
		\subsection{Analyse du nuage de point généré}
		\subsubsection{Outil pour l'analyse de nuage de points: Paraview}
		\todoRediger{Outil de visualisation utilisé: Paraview}
		\todoRediger{besoin de convertir pcd en vtk à l'aide de l'outil approprié}
		\todoRediger{permet de naviguer de manière efficasse dans le nuage de point, d'effectuer des projections, etc, etc}
		
		\subsubsection{Points à améliorer sur le nuage de points}
		\todoRediger{Le nuage est épais}
		
		\todoRediger{octomap génère un nuage de point basé sur le centre des voxel occupés --> problème de quantification volumique implique perte de précision de la mesure}
		
		\todoRediger{Soit réduire la taille du voxel --> augmentation de la taille}
		
		\todoRediger{Soit retourne le point moyen de chaque voxel --> non supporté par octomap --> à implémenter nous même.}
		
		\todoAjouterRef{octomap}
\chapter{Mise en oeuvre à partir de mesures réelles}

Les chapitres précédents ont montrés comment le système Symeter V2 a été initialement conçu et monté à l'aide d'outils de simulation. Après avoir obtenu une première série de résultat par ce biais, quelques questions ont émergé concernant la validité de certains aspects de la simulation, et de savoir si la simulation représentait un chemin réaliste de développement.

\para C'est pour cela qu'une campagne d'acquisation de données réelle, à l'aide des capteurs physiques réels a été effectuée.


	\section{Protocole de test}
	\subsection{Instruments d'acquisition)}
	\todoRediger{TellusCar avec Instrument montés sur l'arrière}
	\todoRediger{données GPS, données IMU et données LIDAR, toutes acquises en même temps}
	\missingfigure{text}
	
	\subsection{Exploitation}
	\todoRediger{Capture Bagfile alors que la TellusCar évolue dans le parking de adjacent à TEllusEnv}
	\todoRediger{La partie logiciel de Symeter V2 peut exploiter des rejeux des captures pour estimer l'évolution de la TellusCar, et tenter de reconstituer le parking.}
	
	\section{Données générée - visualisation sous google maps}
	En rejouant le bagfile nous pouvons extraire et visualiser les données brutes.
	\missingfigure{trajectoire sous Googlemap}
	\missingfigure{graphe des accélerations et de l'assiette issues de l'IMU}
	\missingfigure{données LIDAR brutes}
	
	\section{Exploitation des données}
	\subsection{Configuration: description des poses des instruments par rapport au repère du véhicule}
	
	\subsection{Intégration des données GPS et IMU pour localisation}
	\todoRediger{Il a été compliqué de faire fonctionner la partie localisation de Symeter V2 avec les données réelle}
	\todoRediger{grande sensibilité des données IMU si la pose de l'IMU n'est pas rigoureusement décrite dans la configuration de symeter V2}
	\todoRediger{Obligé de traiter les données IMU pour rétablir la bonne orientation}
	\todoRediger{L'imu mesure le cap avec 0 au nord alors que robot\_localization s'attend à un 0 pour un cap à l'Est}
	
	\subsection{Reconstitution du parking}
	
	\todoRediger{une fois le système de localisation OK, pb avec Octomap qui n'a pas l'air de fonctionner au dela d'une certaine distante}
	
	\section{Récapitulation}
	
	\todoRediger{Illustre les problématiques du passage de la simulation à un integration en grandeur}
	\todoRediger{IMU simulé n'est pas fiable: il faudra en trouver un autre}
	\todoRediger{A l'heure actuelle pas encore terminé}

\chapter{Reste à faire et Axes Améliorations}

\todoPeutEtre{caméra video pour odométrie visuelle}

\section{Difficulté de Simulation de l'action de tassage}
La simulation peut permettre la mise au point de la localization, de l'acquisition initiale du chantier d'ensilage. Mais elle n'est pas directement utilisable pour simuler l'opération du tracteur dans le chantier d'ensilage. En effet pas possible de simuler physiquement le tassage à l'aide de Gazebo.

\para Il est cependant possible de mettre en oeuvre nos outils de manière à simuler logiquement l'évolution du chantier en implémentant un mécanisme de sauvegarde / restauration de l'état du chantier

\todoRediger{Decrire le cycle de simulation}
\missingfigure{Diagramme d'état simulation de l'acte de tassage.}


\chapter{Conclusion}

\begin{appendix}
	\chapter{Positionnement En Robotique}
		\section{Géometrie projective, Coordonnées Homogènes}
		\section{Une autre descriptions des rotations en 3D: Quaternions Unitaires}
		\section{Application: Simulation de couverture d'un faiseau LIDAR orienté vers le sol}
	
	\chapter{Filtres de Kalman}
	
	\chapter{ROS: Architecture et Concepts}
	
	\chapter{Point Cloud Library}
	Essai de citation \cite{kaplan_understanding_2006}.

\end{appendix}

\nocite{*}
\bibliographystyle{alpha}
\bibliography{biblio}

\end{document}