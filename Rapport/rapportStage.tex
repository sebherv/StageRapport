\documentclass[12pt,a4paper]{report}
\usepackage{natbib}
\usepackage[utf8]{inputenc}
\usepackage[french]{babel}
\usepackage[T1]{fontenc}
\usepackage{amsmath}
\usepackage{amsfonts}
\usepackage{amssymb}
\usepackage{graphicx}
\usepackage{appendix}
\usepackage{enumitem}
\usepackage{float} 
\usepackage[nottoc, notlof, notlot]{tocbibind}
\usepackage{pdfpages}
\usepackage{gensymb}
\author{Sébastien Hervieu}
\begin{document}

\begin{titlepage}

\newcommand{\HRule}{\rule{\linewidth}{0.5mm}} % Defines a new command for the horizontal lines, change thickness here

\center % Center everything on the page
 
%----------------------------------------------------------------------------------------
%	HEADING SECTIONS
%----------------------------------------------------------------------------------------

\textsc{\LARGE Université de Rennes 1}\\[1cm] 
\textsc{\Large }\\[0.5cm] % Major heading such as course name
\textsc{\large Master 2 Calcul scientifique et modélisation}\\
\textsc{Rapport de Projet de Préstage}\\
%----------------------------------------------------------------------------------------
%	TITLE SECTION
%----------------------------------------------------------------------------------------

\HRule \\[0.4cm]
{ \huge \bfseries Etude et développement d’outils mathématiques pour estimer, en temps réel, le tassage et le volume d’un silo de maïs à partir de capteurs embarqués}\\[0.4cm] 
\HRule \\[1.5cm]
 
%----------------------------------------------------------------------------------------
%	AUTHOR SECTION
%----------------------------------------------------------------------------------------

\begin{minipage}{0.4\textwidth}
\begin{flushleft} \large
\emph{Auteur:}\\
Sébastien \textsc{Hervieu}
\end{flushleft}
\end{minipage}
~
\begin{minipage}{0.4\textwidth}

\begin{flushright}
	\emph{Tuteur de Stage:} \\
	Geoffroy \textsc{Etaix}
\end{flushright}

\begin{flushright} \large
	\emph{Tuteur Universitaire:} \\
	Fabrice \textsc{Mahé} 
\end{flushright}

\end{minipage}\\[1cm]

% If you don't want a supervisor, uncomment the two lines below and remove the section above
%\Large \emph{Author:}\\
%John \textsc{Smith}\\[3cm] % Your name

%----------------------------------------------------------------------------------------
%	DATE SECTION
%----------------------------------------------------------------------------------------

{\large \today}\\[1cm] % Date, change the \today to a set date if you want to be precise

%----------------------------------------------------------------------------------------
%	LOGO SECTION
%----------------------------------------------------------------------------------------
\includegraphics[height=3cm]{img/logo-tellusenv.png} \\
\includegraphics[height=3cm]{img/univ.jpeg}\\[1cm] % Include a department/university logo - this will require the graphicx package
 
%----------------------------------------------------------------------------------------

\vfill % Fill the rest of the page with whitespace

\end{titlepage}

\tableofcontents
\newpage

\chapter{Introduction}
\section{Tellus Environment}
Activité: Géophyisique
Activité: R\&D: développer des composants qui mettent en oeuvre les expertises de TellusEnvironment pour créer des produits innovants, dont les composants seraient de plus réutilisables pour améliorer la productivité des services Géophysique.
Note finale: aller sur le terrain

\chapter{Contexte projet}
Symeter V1: rapide rappel
Symeter V2: Objectifs
Symeter V2: Présentation du plan de projet
- test du simulateur Gazebo pour évaluer son utilité dans le projet Symeter2
- Simulation couverture lidar
- Montage des outils nécessaires au développement simulé du projet symeter
- Mise en place de la localisation: installation et tests
- Mise en place de l'acquisition des relevés.


\chapter{Simulation de couverture d'un faiseau LIDAR orienté vers le sol}

\chapter{Simulation robotique en utilisant ROS/Gazebo}

\chapter{Mise en place du processus de localisation}

\chapter{Processus d'acquisition des relevés à base de LIDAR}

\begin{appendix}
	\chapter{Géometrie projective, Coordonnées Homogènes}
	
	\chapter{Filtres de Kalman}
	
	\chapter{ROS: Architecture et Concepts}
	
	\chapter{Point Cloud Library}

\end{appendix}

\bibliographystyle{plain}
\bibliography{biblio}

\end{document}